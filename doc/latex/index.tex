\section*{neuro-\/1}

\subsection*{P\+R\+O\+J\+E\+CT IN N\+E\+U\+R\+O\+S\+C\+I\+E\+N\+CE, G\+R\+O\+UP N\+E\+U\+R\+O-\/1}

\subsubsection*{I\+N\+T\+R\+O\+D\+U\+C\+T\+I\+ON}

This project aims to simulate a cortical neural network. The network consists of 12500 neurons, of which 20\% are inhibitory and 80\% are excitatory. The probability of a connection from one neuron to another is 10\%. In addition, all neurons are connected to background neurons outside the network. The \char`\"{}background noise\char`\"{} generated by these is simulated by a Poisson distribution.

The behavior of the neurons in our network is based on a differential equation describing the change in the neuron\textquotesingle{}s membrane potential over time. When the potential of a neuron reaches a certain threshold value, it will send a spike to the neurons it is connected to.

Our program produces M\+A\+T\+L\+AB graphs\+: The first graph displays the spikes of 50 randomly chosen neurons, and the second graph displays the number of spikes of the entire network, both as a function of time.

\subsubsection*{B\+U\+I\+L\+D\+I\+NG}

Place yourself in the build directory and execute the following commands in the terminal\+: \begin{DoxyVerb}    cmake ..
    make
\end{DoxyVerb}


\subsubsection*{R\+U\+N\+N\+I\+NG T\+HE S\+I\+M\+U\+L\+A\+T\+I\+ON}

To run the simulation, the user needs to add the flag (\char`\"{}-\/r 1\char`\"{}) when running the executable generated when building.

\paragraph*{S\+E\+T\+T\+I\+NG P\+A\+R\+A\+M\+E\+T\+E\+RS}

The user can modify the parameters of the program while executing it in the terminal. If the value of a particular parameter is not specified, its default value will be used.

Flags, to be followed by their desired value\+:

\char`\"{}-\/f\char`\"{}\+: the ratio vext/vthr, which will be used to calculate the external input frequency. Default\+: 5 \char`\"{}-\/g\char`\"{}\+: the relative amplitude g of inhibitory neurons (versus excitatory). Default\+: 2 \char`\"{}-\/j\char`\"{}\+: the excitatory amplitude Je (in m\+Volt), the amplitude of a spike from an inhibitory neuron. Default\+: 0.\+1 \char`\"{}-\/h\char`\"{}\+: get further information, e.\+g. default values of the parameters

Each parameter can take positive floating point or integer values. If you provide an unacceptable value, the program will terminate and you will have to try again with another value.

\paragraph*{E\+X\+A\+M\+P\+LE E\+X\+E\+C\+U\+T\+I\+O\+NS}

\subparagraph*{Example execution 1\+:}

\begin{DoxyVerb}    ./Neuronsimulation -r 1 -g 4.0 -j 0.3
\end{DoxyVerb}


In this example, the program will execute with the default value of the ratio Vext/\+Vthr, a relative inhibitory amplitude of 4.\+0 and an excitatory amplitude of 0.\+3.

\subparagraph*{Example execution 2\+:}

\begin{DoxyVerb}    ./Neuronsimulation -r 1
\end{DoxyVerb}


In this example, the program will execute with the default values of all the parameters. These correspond to the graph C in Brunel 2000.

\subsubsection*{G\+E\+N\+E\+R\+A\+T\+I\+NG G\+R\+A\+P\+HS}

Running the simulation as described above will generate two .txt files in the build folder\+: spikes.\+txt and sum\+\_\+spikes.\+txt. To generate the graphs described in the introduction, M\+A\+T\+L\+AB is necessary. The M\+A\+T\+L\+AB scripts in the matlab/ folder are scaled to the respective model in Brunel, depending on the parameters the program was run with\+:

Model A\+: g = 3, vext/vthr = 2, Je = 0.\+1 --$>$ use script\+\_\+graphs\+A.\+m Model B\+: g = 6, vext/vthr = 4, Je = 0.\+1 --$>$ use script\+\_\+graphs\+B.\+m Model C\+: g = 5, vext/vthr = 2, Je = 0.\+1 --$>$ use script\+\_\+graphs\+C.\+m Model D\+: g = 4.\+5, vext/vthr = 0.\+9, Je = 0.\+1 --$>$ use script\+\_\+graphs\+D.\+m

In M\+A\+T\+L\+AB\+: Open the respective script by selecting \char`\"{}\+Open\char`\"{} and then choosing the script from the neuro-\/1/matlab folder. (If M\+A\+T\+L\+AB warns you that it the script is not found in the current folder, select \char`\"{}\+Change Folder\char`\"{}.) To generate the respective graph, select \char`\"{}\+Run\char`\"{}. The upper part of the graph shows the spikes per time step of 50 random neurons, the lower part shows the sum of spikes of all neurons per time step.

\subsubsection*{G\+E\+N\+E\+R\+A\+T\+I\+NG D\+O\+C\+U\+M\+E\+N\+T\+A\+T\+I\+ON}

Place yourself in the build directory and execute the following commands in the terminal\+: \begin{DoxyVerb}    cmake ..
    make
    make doc
\end{DoxyVerb}


Then open neuro-\/1/doc/html/index.\+html in a browser. This is the documentation page of this project.

\subsubsection*{R\+U\+N\+N\+I\+NG T\+E\+S\+TS}

Place yourself in the build directory and execute the following commands in the terminal\+: \begin{DoxyVerb}    cmake ..
    make
    make test
\end{DoxyVerb}


This generates an executable, Neuron\+Simulation\+\_\+test, which can be launched for further information. Make tests indicates whether the test passed.

\subsubsection*{C\+O\+N\+T\+R\+I\+B\+U\+T\+O\+RS}

Gaia Carparelli ~\newline
 Alexandre Horii-\/\+Huber~\newline
 Sophie Jolidon~\newline
 Lucas Eckes~\newline
 Hugo Michel~\newline
 Grégory Tristan Mathez~\newline
 Delberghe Florian~\newline
 Elia Fernandez~\newline
 Lena Bruhin~\newline
 Linn Hornwall ~\newline
 Emile Bourban

\subsubsection*{R\+E\+F\+E\+R\+E\+N\+CE}

Brunel Nicolas (2000), Dynamics of Sparsely Connected Networks of Excitatory and Inhibitory Spiking Neurons, Journal of Computational Neuroscience 8, 183–208. 